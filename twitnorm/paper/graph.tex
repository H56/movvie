\subsection{Graph Based Contextual Similarity}

Given a entry to normalize, next step is extracting normalization candidates for each OOV token with contextual similarity features. For each ill-formed OOV token $t_i$ in a given entry, we start with a list of related nodes of $t_i$ within that entry. The list includes all the nodes related to $t_i$ and their positional distance to the $t_i$. We will refer this list as neighbour list $NL_i$. In Table~\ref{tab:neigh} you can find a sample neighbour list for the OOV token beautiful$|$A from the sample sentence in Table~\ref{tab:graph}.

\begin{table}[hbt]
  \centering
  \begin{tabular}[tc]{l}
    w$|$P, position: -2 \\
    a$|$D, position: -1 \\
    smile$|$V, position: 1 \\
  \end{tabular}
\caption{Example neighbour list for the beautiful$|$A}
\label{tab:neigh}
\end{table}


For each neighbour node $n_{iJ}$ in the $NL_i$  we traverse the graph and find the edges from or to the node ($n_{iJ}$). The resulting edge list $EL_{ij}$ has edges in the form of ($n_{iJ}$, candidate) or (candidate,$n_{iJ}$) and includes all the nodes that are related to the node $n_{iJ}$. Here the neighbour node can be an OOV node but the candidate node chosen among the IV nodes. We filter the edges in $EL_{ij}$ by the relative distance of $n_{iJ}$ to the $t_i$ given in the $NL_i$. Each (neighbour, candidate) tuple should have the exact distance as the neighbour and OOV token has.

One last step is to conduct a POS tag filtering on the edges in $EL_{ij}$. Each candidate node should have the same POS tag with its OOV token. For the OOV token $t_i$ which has tag $T_i$, all the edges that includes candidates with a tag other than $T_i$ is removed from the edge list $EL_{ij}$. Thus $EL_{i}$ contains edges only with $c_{i}$s that is tagged with $T_i$.

Each edge in the $EL_{ij}$ has a neighbour node $n_{iJ}$, a candidate node $c_{iJ}$ and an edge weight $ew_{iJ}$. The edge weight, represents the likelihood or the degree of the relatedness between the the neighbour node $n_{iJ}$ and candidate node $c_{iJ}$. Since we are looking for candidate nodes that is most likely to be the correct canonical form of the OOV word $t_i$, this metric is a good indicator. However weight of the common phrases are much more higher than the average weight. That results in favouring the words with higher frequencies like stop words. To avoid this we normalize the edge weight $ew_{iJ}$ with the frequency of the neighbour node $n_{iJ}$.

Elimizde her komşunun binary relatedness scoreları var. Genel bir contextual similarity metriği elde edebilmek için binary relatedness'tan daha fazlasına ihtiyacımız var. Bu yüzden list of edge list'leri dolaşıp her aday için farklı komşular tarafından verilen normalized weight metriğini toplayarak contextual similarity feature'ımız çıkarıyoruz/elde ediyoruz. bu feauture'ın en çok komşu ile ençok binary related olan adayları öne çıkarmasını umuyoruz.

Aday node'ların frekanslarını normalize ederek frequency metric'i oluşturduk. Relatedness graph'ımızda 

Genel contextual similarity metrik = sum_of_binary_relatedness_coming_from_each_neighbour + frequency metric

Böylece elimizde $EL_{ij}$'teki unique adayları ve contextual sim score'larıyla candidate list $CL_{ij}$ oluşturuyoruz.

\subsection{Lexical Similarity}

Following ~\cite{Han:2011:LNS:2002472.2002520},\cite{DBLP:conf/acl/HassanM13}, we built our lexical similarity features over standart edit distance~\cite{levenshtein1966bcc}, double metaphone(phonetic edit distance)~\cite{Philips:2000:DMS:349124.349132} and longest common subsequence ratio over edit distance(LCRS)~\cite{Contractor:2010:UCN:1944566.1944588}.

We calculated both edit distance, phonetic edit distance and LCSR of each candidate in $CL_{ij}$ and OOV token $t_i$.

We used edit distance, phonetic edit distance to filter the candidates. Any candidate in $CL_{ij}$ with an edit distance greater than $ed_t$ and phonetic edit distance greater than $ped_t$ to $t_i$ has been removed from the candidate list $CL_{ij}$.

For the rest of the candidates we calculated LCSR and edit distance measure that is between 0.0 and 1.0. (*https://sourceforge.net/projects/febrl/ ) 

İkisini toplayarak sim_cost'umuzu bulduk.

Additionally, sistemin kısa entry'ler(context bağlantısı kurulamayacak kadar kısa), ve context'ten bağımsız token'ları da kapsayabilmesi için graph'ı tekrar dolaşıp yukarıdaki filtreye uyan kelimeleri de aday listesine ekledik.

son olarak slang dictionaryden faydalanarak aday yelpazemizi genişlettik.

\subsection{Ranking}

butun parametetreler aynı.Hassan'ların lambdası
