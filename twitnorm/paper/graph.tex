\subsection{Lexical Similarity}

Following~\cite{Han:2011:LNS:2002472.2002520},\cite{DBLP:conf/acl/HassanM13}, we built our lexical similarity features over standard edit distance~\cite{levenshtein1966bcc}, double metaphone(phonetic edit distance)~\cite{Philips:2000:DMS:349124.349132} and longest common subsequence ratio over edit distance(LCSR)~\cite{Contractor:2010:UCN:1944566.1944588}.

We calculated both edit distance, phonetic edit distance and LCSR of each candidate in $CL_{ij}$ and OOV token $t_i$.

Following the tradition that is inspired from~\cite{Kaufmann2010} before lexical similarity calculations any repetitions more than 3 letters in OOV tokens are reduced to 2. (gerci onlar 3e reduce etmis ama)

We used edit distance and phonetic edit distance to filter the candidates. Any candidate in $CL_{ij}$ with an edit distance greater than $ed_t$ and phonetic edit distance greater than $ped_t$ to $t_i$ has been removed from the candidate list $CL_{ij}$.

For the rest of the candidates we calculated the total lexical similarity score(Eq~\ref{eq:lexscore}) using LCSR and edit distance score~\footnote{an approximate string comparator measure (between 0.0 and 1.0) using the edit distance \url{https://sourceforge.net/projects/febrl/}}. Since the main lexical feature is LCSR, we applied the same distributions we used in contextual similarity score in calculating lexical score too.

\begin{equation}
lexScore(t,c) = \lambda_a LCSR(t,c) + \frac{\lambda_a} 2 editDistScore(t,c) + \lambda_a externalScore(t,c)
\label{eq:lexscore}
\end{equation}

Additionally, sistemin kısa entry'ler(context bağlantısı kurulamayacak kadar kısa), ve context'ten bağımsız token'ları da kapsayabilmesi için graph'ı tekrar dolaşıp contextual similarity feature'ı taşımasa da yukarıdaki filtreye uyuyorsa aday listesine ekledik.


son olarak external normalizasyonlar olarak slang dictionarydeki entryleri, sayıları ve pronounları externalScore veretek aday listesine ekliyoruz. externalScore binary bir feature 0 ya da 1.
\begin{verbatim}
units = ["", "one", "to", "three", "for",  "five", "six", "seven", "eight", "nine"]
pronouns = {u'2':u"to",u'w':u"with"}
\end{verbatim}


\begin{equation}
candScore(t,c) = lexScore(t,c) + contScore(t,c)
\label{eq:candscore}
\end{equation}
