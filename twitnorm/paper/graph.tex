\subsection{Graph Based Contextual Similarity}

Given a entry to normalize, next step is extracting normalization candidates for each OOV token with contextual similarity features. For each ill-formed OOV token in a given entry, we start with listing the related tokens in that entry. The list includes all the related words of an ill-formed OOV token and their distance to the OOV token. We will refer this list as neighbour list. In Table~\ref{tab:neigh} you can find a sample neighbour list for the OOV token beautiful$|$A from the sample sentence in Table~\ref{tab:graph}.

\begin{table}[hbt]
  \centering
  \begin{tabular}[tc]{l}
    w$|$P, distance: 2 \\
    a$|$D, distance: 1 \\
    smile$|$V, distance: 1 \\
  \end{tabular}
\caption{Example neighbour list for the beautiful$|$A}
\label{tab:neigh}
\end{table}


For each neighbour in the neighbour list we traverse the graph and find the edges from or to the neighbour. These edges (neighbour,cand) or

Candidate selection from graph

\subsection{Lexical Similarity}

dictionary from graph

slang dictionary lookup

double metaphone 1
edit distance 2


longest common sub-sequence ratio



\subsection{Ranking}

butun parametetreler aynı.Hassan'ların lambdası
