


\subsection{Graph construction}

\subsection{Graph Based Contextual Similarity}


Candidate selection from graph

\subsection{Lexical Similarity}

dictionary from graph

slang dictionary lookup

double metaphone 1
edit distance 2


longest common sub-sequence ratio



\subsection{Ranking}

butun parametetreler aynı.Hassan'ların lambdası

\begin{table}[tbhp]
\begin{centering}
\caption{Example noisy n-grams}
\label{tab:ngrams}
\begin{tabular}[h]{l}
\hline
with a beautiful smile \\
\hline
with a \textbf{beatiful} smile \\
\textbf{w} a beautiful smile \\
\textbf{wit} a beautiful \textbf{smil} \\
\textbf{wth} a \textbf{btfl} \textbf{sml} \\
\textbf{w} a \textbf{btfl smle} \\
\hline
\end{tabular}
\par\end{centering}
\end{table}





We tried to overcome this by building a distance aware co-occurrence representation of an n-gram language model. We builded a  graph, nodes representing the tokens and edges representing the co-occurrence relation between tokens.

Each node includes a token and tokens' POS tag.

A node consists of four properties \textit{id, ovv, freq, tag}. The token itself plus it's POS tag forms the \textit{id} field. Each token is represented with it's part of speech tag, this helps us to distinguish the different beings of tokens within the texts, helps not to propose smile as a pronoun. \textit{freq} property indicates the node's frequency count in the dataset and OOV field set to True if a token is OOV. Following Han et al. we used GNU Aspell dictionary (v0.60.6) to determine whether a word is OOV.

\begin{verbatim}
 {u'_id': u'smile|A', u'freq': 3, u'oov': False, u'tag': u'A'}
 {u'_id': u'smile|N', u'freq': 3403, u'oov': False, u'tag': u'N'}
 {u'_id': u'smile|V', u'freq': 2796, u'o0v': False, u'tag': u'V'},
\end{verbatim}

The edges graph, represents the context of tweets. It is the co-occurrence information including the distance information between words. For example the edges below would be derived from a text including the phrase ``with a beautiful smile''. The \textit{from} property indicates the first word and \textit{to} is the latter in the co-occurrence. Each co-occurrence of two words increases the weight of the representing edge with the average of their POS tag accuracy score in that specific text. If we are to expand the graph with our example phrase with the given POS tags and accuracies below. The increase in the weights would be respectively $0.9963+0.9712/2$, $0.998+0.9712/2$ and $0.9971+0.9712/2$.

\begin{verbatim}
{ "dis" : 3, "from" : "with|P", "to" : "smile|N", "weight" : 10.47095 }
{ "dis" : 1, "from" : "a|D", "to" : "smile|N", "weight" : 274.37365 }
{ "dis" : 0, "from" : "beautiful|A", "to" : "smile|N", "weight" : 240.716 }
\end{verbatim}



Our model makes use of words up to distance 4 to represent an equal form of a 5-gram language model.
